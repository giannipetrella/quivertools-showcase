% Layout, characters, page margins, pictures handling.
\usepackage{graphicx, float}                        % Images
\usepackage{colonequals}                            % poor man's \colonequals
% Math tools
\usepackage{amsmath, amsthm, amsfonts, amssymb}     % A lot of symbols

% References and hyperlinks both in the document and to the web. Needs two compilations in a row.
\usepackage{hyperref, bookmark}

% Commutative diagrams
\usepackage{tikz, tikz-cd}
\usetikzlibrary{calc} % This allows for slightly moving nodes in a tikzpicture, by using ($(m-1-1.east)+(offsetX,offsetY)$) for example.
\usetikzlibrary{arrows, matrix} % I guess

% \usepackage{fullpage}

% % typesetting
% \usepackage[T1]{fontenc}
% \usepackage{libertine}
% \usepackage[libertine]{newtxmath}
% \usepackage[scaled=0.83]{beramono}
% \usepackage{parskip}                                % To handle paragraph spacing, indentation etc.
% %\usepackage[charter]{mathdesign}
% %\usepackage[scaled]{beramono,berasans}
% \usepackage{eucal}
% \usepackage{microtype}
% \frenchspacing

%%%%%%%%%%%%%%%%%%%%%%%%%%%%%%%%%%%%%%%%%%%%%%%%%%%%%%%%%%%%%%%%%%%%%%%%%%%%%%%%%%%%%%%%%%%%%%%%%%%
% Bibliography management. Compile with biber.
% \usepackage[backend=biber, datamodel=mrnumber, sortcites]{biblatex}
% \addbibresource{bibliography.bib}

\usepackage{gitinfo2}

\newcommand\gitfootnote[1]{% yoinked from Pieter's repo
  \begin{NoHyper}
  \renewcommand\thefootnote{}\footnote{#1}%
  \addtocounter{footnote}{-1}%
  \end{NoHyper}
}

%%%%%%%%%%%%%%%%%%%%%%%%%%%%%%%%%%%%%%%%%%%%%%%%%%%%%%%%%%%%%%%%%%%%%%%%%%%%%%%%%%%%%%%%%%%%%%%%%%%
% Theorems are numbered within sections using a common counter. Do away with ambiguous numbering schemes and assign A number to A thing.
\theoremstyle{plain}
% % \newtheorem{theorem}{Theorem}[section]
% \newtheorem{proposition}[theorem]{Proposition}
% \newtheorem{lemma}[theorem]{Lemma}
% \newtheorem{corollary}[theorem]{Corollary}
% \newtheorem{claim}[theorem]{Claim}
\newtheorem{conjecture}{Conjecture}

% \theoremstyle{remark}
% \newtheorem{remark}[theorem]{Remark}

% \theoremstyle{definition}
% \newtheorem{definition}[theorem]{Definition}
% \newtheorem{example}[theorem]{Example}



%%%%%%%%%%%%%%%%%%%%%%%%%%%%%%%%%%%%%%%%%%%%%%%%%%%%%%%%%%%%%%%%%%%%%%%%%%%%%%%%%%%%%%%%%%%%%%%%%%%%


\newcommand{\iif}{\ensuremath{\Leftrightarrow}}              % If and only if

\renewcommand*{\to}[1][]{\overset{#1}{\rightarrow}} % Arrow with optional label. Use as A \to[label] B


\newcommand{\stable}{-\mathrm{st}}
\newcommand{\semistable}{-\mathrm{sst}}


\newcommand{\dirlim}[1]{                        % Direct limit
    \varinjlim_{\substack{#1}}
    }

\newcommand{\invlim}[1]{                        % Inverse limit
    \varprojlim_{\substack{#1}}
    }

\DeclareMathOperator{\Hom}{Hom}
\DeclareMathOperator{\dom}{dom}
\DeclareMathOperator{\im}{Im}
\DeclareMathOperator{\GL}{GL}
\DeclareMathOperator{\Mat}{Mat}

% Numbers
\newcommand{\NN}{\mathbb{N}}
\newcommand{\ZZ}{\mathbb{Z}}
\newcommand{\QQ}{\mathbb{Q}}
\newcommand{\RR}{\mathbb{R}}
\newcommand{\CC}{\mathbb{C}}

% Spaces
\renewcommand{\AA}{\mathbb{A}}
\newcommand{\PP}{\mathbb{P}}


% Sheaves

\DeclareMathOperator{\sheafHom}{\mathscr{H}\kern -2.5pt\mathit{om}}
\DeclareMathOperator{\sheafExt}{\mathscr{E}\mathit{xt}}

% various
\newcommand{\tuple}[1]{\mathbf{#1}}
\newcommand{\gianni}[1]{{\color{blue}#1}}

% macros
\DeclareMathOperator{\parameterspace}{R}
\DeclareMathOperator{\modulispace}{M}
\DeclareMathOperator{\group}{\GL_{\tuple{d}}}

\DeclareMathOperator{\HH}{H}