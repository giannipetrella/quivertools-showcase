\documentclass{beamer}
% Layout, characters, page margins, pictures handling.
\usepackage{graphicx, float}                        % Images
\usepackage{colonequals}                            % poor man's \colonequals
% Math tools
\usepackage{amsmath, amsthm, amsfonts, amssymb}     % A lot of symbols

% References and hyperlinks both in the document and to the web. Needs two compilations in a row.
\usepackage{hyperref, bookmark}

% Commutative diagrams
\usepackage{tikz, tikz-cd}
\usetikzlibrary{calc} % This allows for slightly moving nodes in a tikzpicture, by using ($(m-1-1.east)+(offsetX,offsetY)$) for example.
\usetikzlibrary{arrows, matrix} % I guess

% \usepackage{fullpage}

% % typesetting
% \usepackage[T1]{fontenc}
% \usepackage{libertine}
% \usepackage[libertine]{newtxmath}
% \usepackage[scaled=0.83]{beramono}
% \usepackage{parskip}                                % To handle paragraph spacing, indentation etc.
% %\usepackage[charter]{mathdesign}
% \usepackage[scaled]{beramono,berasans}
% \usepackage{eucal}
% \usepackage{microtype}
% \frenchspacing

%%%%%%%%%%%%%%%%%%%%%%%%%%%%%%%%%%%%%%%%%%%%%%%%%%%%%%%%%%%%%%%%%%%%%%%%%%%%%%%%%%%%%%%%%%%%%%%%%%%
% Bibliography management. Compile with biber.
% \usepackage[backend=biber, datamodel=mrnumber, sortcites]{biblatex}
% \addbibresource{bibliography.bib}

\usepackage{gitinfo2}

\newcommand\gitfootnote[1]{% yoinked from Pieter's repo
  \begin{NoHyper}
  \renewcommand\thefootnote{}\footnote{#1}%
  \addtocounter{footnote}{-1}%
  \end{NoHyper}
}

%%%%%%%%%%%%%%%%%%%%%%%%%%%%%%%%%%%%%%%%%%%%%%%%%%%%%%%%%%%%%%%%%%%%%%%%%%%%%%%%%%%%%%%%%%%%%%%%%%%
% Theorems are numbered within sections using a common counter. Do away with ambiguous numbering schemes and assign A number to A thing.
\theoremstyle{plain}
% % \newtheorem{theorem}{Theorem}[section]
% \newtheorem{proposition}[theorem]{Proposition}
% \newtheorem{lemma}[theorem]{Lemma}
% \newtheorem{corollary}[theorem]{Corollary}
% \newtheorem{claim}[theorem]{Claim}
\newtheorem{conjecture}{Conjecture}

% \theoremstyle{remark}
% \newtheorem{remark}[theorem]{Remark}

% \theoremstyle{definition}
% \newtheorem{definition}[theorem]{Definition}
% \newtheorem{example}[theorem]{Example}



%%%%%%%%%%%%%%%%%%%%%%%%%%%%%%%%%%%%%%%%%%%%%%%%%%%%%%%%%%%%%%%%%%%%%%%%%%%%%%%%%%%%%%%%%%%%%%%%%%%%


\newcommand{\iif}{\ensuremath{\Leftrightarrow}}              % If and only if

\renewcommand*{\to}[1][]{\overset{#1}{\rightarrow}} % Arrow with optional label. Use as A \to[label] B


\newcommand{\stable}{-\mathrm{st}}
\newcommand{\semistable}{-\mathrm{sst}}


\newcommand{\dirlim}[1]{                        % Direct limit
    \varinjlim_{\substack{#1}}
    }

\newcommand{\invlim}[1]{                        % Inverse limit
    \varprojlim_{\substack{#1}}
    }

\DeclareMathOperator{\Hom}{Hom}
\DeclareMathOperator{\Ext}{Ext}
\DeclareMathOperator{\dom}{dom}
\DeclareMathOperator{\im}{Im}
\DeclareMathOperator{\GL}{GL}
\DeclareMathOperator{\Mat}{Mat}

% Numbers
\newcommand{\NN}{\mathbb{N}}
\newcommand{\ZZ}{\mathbb{Z}}
\newcommand{\QQ}{\mathbb{Q}}
\newcommand{\RR}{\mathbb{R}}
\newcommand{\CC}{\mathbb{C}}

% Spaces
\renewcommand{\AA}{\mathbb{A}}
\newcommand{\PP}{\mathbb{P}}


% Sheaves

\DeclareMathOperator{\sheafHom}{\mathscr{H}\kern -2.5pt\mathit{om}}
\DeclareMathOperator{\sheafExt}{\mathscr{E}\mathit{xt}}

% various
\newcommand{\tuple}[1]{\mathbf{#1}}
\newcommand{\gianni}[1]{{\color{blue}#1}}

% macros
\DeclareMathOperator{\parameterspace}{R}
\DeclareMathOperator{\modulispace}{M}
\DeclareMathOperator{\group}{\GL_{\tuple{d}}}

\DeclareMathOperator{\HH}{H}

\title{Decomposing quiver moduli - a QuiverTools showcase}
\author{Gianni Petrella - University of Luxembourg}
\institute{MEGA 2024 - MPI MIS Leipzig}
\date{July 29th, 2024}


\begin{document}
\begin{frame}
    \titlepage
% \footnotetex{commit: \texttt{\gitAbbrevHash}\hfil date: \texttt{\gitAuthorIsoDate}\hfil \texttt{\gitReferences}}
\end{frame}

\begin{frame}
    \frametitle{Preamble}
\textbf{Plan}
\begin{enumerate}
    \item What are quiver moduli?
    \item What is \emph{QuiverTools}?
    \item $D^b(Q)$ vs $D^b(M)$
    \item Semiorthogonal embeddings
\end{enumerate} \pause

\textbf{Aknowledgements}

This work is supported by the Luxembourg National Research Fund (AFR-17953441)

\end{frame}

\begin{frame}
    \frametitle{Quiver representations}
\begin{definition}
    We work over $\mathbb{C}$. \pause

    A \emph{quiver}~$Q$ is a finite directed graph, with
    vertices~$Q_0$ and arrows~$Q_1$.

    A \emph{representation}~$V$ of~$Q$ is
    a choice of a vector space~$V_i$ per vertex~$i$
    and of a linear map~$V_{\alpha}$ per arrow~$\alpha$.
\end{definition} \pause

Once a \emph{dimension vector} $\tuple{d}$ is fixed, a representation
is identified with a point in the \emph{parameter space}
\[
    \parameterspace(Q, \tuple{d})
    \colonequals \bigoplus_{i \to j \in Q_1}
    \Mat_{d_j \times d_i}(k).
\] \pause

The group~${\GL}_{\tuple{d}} \colonequals \bigoplus_{i \in Q_0} \GL_{d_i}$
acts on~$\parameterspace$ by base change. \pause

Orbits are precisely isomorphism classes of representations.
% and two points $V, W$ belong to the
% same orbit precisely when the corresponding representations are isomorphic. \pause
\end{frame}

\begin{frame}
    \frametitle{Quiver moduli via GIT}
A \emph{stability parameter} is~$\theta \in \ZZ^{Q_0}$
for which~$\theta \cdot \tuple{d} = 0$. \pause
\begin{definition}
The representation~$V$ is (semi)\emph{stable}
if all its proper subrepresentations~$W$ satisfy~$\theta \cdot W (\leq)< 0$.
\end{definition} \pause
\begin{theorem}
The (semi)stable locus~$\parameterspace^{(\theta\semistable)\theta\stable}(Q, \tuple{d})$
is a~$\GL_{\tuple{d}}$-invariant Zariski open which admits a geometric quotient,
denoted by~$\modulispace^{(\theta\semistable)\theta\stable}(Q, \tuple{d})$.
\end{theorem} \pause

\textbf{Facts} \pause

\begin{itemize}
    \item $\modulispace^{\theta\semistable}(Q, \tuple{d})$ is projective-over-affine,
    projective if $Q$ is acyclic.
    \item $\modulispace^{\theta\stable}(Q, \tuple{d})$ is smooth.
\end{itemize}
\end{frame}

\begin{frame}
    \frametitle{Who cares?}
\begin{itemize}
    \item Large class of possibly smooth, projective varieties; \pause
    \item Encode linear algebra problems geometrically; \pause
    \item Deep parallels between moduli of vector bundles on curves
        and quiver representations. \pause
    \item Geometric properties are often implementable.
\end{itemize}

    

\end{frame}

\begin{frame}
    \frametitle{QuiverTools}
A package to deal with quivers and moduli of quiver representations. \pause

Available as

\begin{itemize}
    \item \href{quivertools.github.io/QuiverTools}{A SageMath library},
    \item \href{quivertools.github.io/QuiverTools.jl}{A Julia package}.
\end{itemize}\pause
\begin{center}
\includegraphics[width=100pt]{quivertools-logo.png}
\end{center}
\end{frame}

\begin{frame}
    \frametitle{Harder-Narasimhan stratification}
A \emph{slope function} is a function~$\mu_{\theta} : \ZZ^{Q_0} \to \QQ: ~\mu_{\theta}(\alpha) = \frac{\theta \cdot \alpha}{\sum_i \alpha_i}$. \pause
\begin{definition}

A \emph{Harder--Narasimhan type} is a
tuple~$\tuple{d}^* = (\tuple{d}^1,\dots,\tuple{d}^s)$, such
that~$\mu(\tuple{d}^1) < \dots < \mu(\tuple{d}^s)$, each dimension
vector admits a semistable representation,
and~$\sum_{\ell = 1}^{s}\tuple{d}^{\ell} = \tuple{d}$. \pause

Given a stability parameter~$\theta$, every representation $V$
admits a unique \emph{Harder--Narasimhan filtration},
i.e.,~$0 = V_0 \subsetneq V_1 \dots \subsetneq V_s = V$ such that the
dimension vectors $\dim(V_{\ell}/V_{\ell-1})$ form an HN type.
\end{definition} \pause

\begin{theorem}[Reineke]
The paremeter space $\parameterspace$ admits a stratification into
smooth, disjoint, locally closed subsets $S_{\tuple{d}^*}$, each corresponding
to a HN type. The trivial type $(\tuple{d})$ corresponds to the semistable
locus~$\parameterspace^{\theta\semistable}(Q, \tuple{d})$.
\end{theorem}

\end{frame}

\begin{frame}
    \frametitle{HN types in QuiverTools}
TODO
\end{frame}

% \begin{frame}
%     \frametitle{$D^b(Q)$ and $D^b(\modulispace)$.}
% [cite pieter]
% Consider the functor $D^b(Q) \to D^b(M)$.
% \begin{theorem}[Bieter Pelmans]
% under reasonable assumptions, this functor is fully faithful.
% \end{theorem}
% In particular, a copy of $D^b(Q)$ is embedded in $D^b(M)$. \pause

% Following [the parallel between VBAC and QR],
% we attempt to embed multiple copies of $D^b(Q)$
% and try and understand whether they are semiorthogonal.

% \begin{conjecture}
% Let $Q$, $d$ and $\theta$ satisfy some reasonable assumptions.
% Then, there is a partial semiorthogonal decomposition of $D^b(M)$ given by
% \[\langle
% \Phi_{\mathcal{U}}(D^b(Q)), \mathcal{O}_{M},
% \Phi_{\mathcal{U(H)}}(D^b(Q)), \mathcal{O}_{M}(H), \dots
% \Phi_{\mathcal{U}((r-1)H)}(D^b(Q)), \mathcal{O}_{M}((r-1)H), \dots
% \rangle\]
% \end{conjecture} \pause
% \end{frame}

\end{document}


% construction of quiver moduli: from zero to M^sst(Q,d) 5 minutes
% derived categories: 5 minutes
% 